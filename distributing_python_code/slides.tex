\documentclass[12pt,handout]{beamer}
\hypersetup{pdfstartview={Fit}}
\usepackage[T1]{fontenc}
\usepackage{cmap}
\usepackage{hyperref}
\usepackage{listings}
\usetheme{default}
\usefonttheme{professionalfonts}
\usefonttheme{serif}
\usepackage{fontspec}
\setmainfont{Liberation Serif Bold}
\title[Distributing Python Packages Using Setuptools]{Distributing Python Packages Using Setuptools}
\author{Baiju Muthukadan}
\institute{ZeOmega, Bangalore}
\date{Feb 27, 2014}

\setbeamertemplate{frametitle}
  {\begin{centering}\smallskip
   \insertframetitle\par
   \smallskip\end{centering}}
\setbeamertemplate{itemize item}{$\bullet$}
\setbeamertemplate{navigation symbols}{}
\setbeamertemplate{footline}[text line]{%
    \hfill\strut{%
        \scriptsize\sf\color{black!60}%
        \quad\insertframenumber
    }%
    \hfill
}

% Define some colors:
\definecolor{DarkFern}{HTML}{407428}
\definecolor{DarkCharcoal}{HTML}{4D4944}
\colorlet{Fern}{DarkFern!85!white}
\colorlet{Charcoal}{DarkCharcoal!85!white}
\colorlet{LightCharcoal}{Charcoal!50!white}
\colorlet{AlertColor}{orange!80!black}
\colorlet{DarkRed}{red!70!black}
\colorlet{DarkBlue}{blue!70!black}
\colorlet{DarkGreen}{green!70!black}

% Use the colors:
%% \setbeamercolor{title}{fg=black}
%% \setbeamercolor{frametitle}{fg=black}
%% \setbeamercolor{normal text}{fg=black}
%% \setbeamercolor{block title}{fg=black,bg=Fern!25!white}
%% \setbeamercolor{block body}{fg=black,bg=Fern!25!white}
%% \setbeamercolor{alerted text}{fg=AlertColor}
%% \setbeamercolor{itemize item}{fg=Charcoal}

\setbeamercolor{alerted text}{fg=orange}
\setbeamercolor{background canvas}{bg=white}
\setbeamercolor{block body alerted}{bg=normal text.bg!90!black}
\setbeamercolor{block body}{bg=normal text.bg!90!black}
\setbeamercolor{block body example}{bg=normal text.bg!90!black}
\setbeamercolor{block title alerted}{use={normal text,alerted text},fg=alerted text.fg!75!normal text.fg,bg=normal text.bg!75!black}
\setbeamercolor{block title}{bg=blue}
\setbeamercolor{block title example}{use={normal text,example text},fg=example text.fg!75!normal text.fg,bg=normal text.bg!75!black}
\setbeamercolor{fine separation line}{}
\setbeamercolor{frametitle}{fg=brown}
\setbeamercolor{item projected}{fg=black}
\setbeamercolor{normal text}{bg=white,fg=black}
\setbeamercolor{palette sidebar primary}{use=normal text,fg=normal text.fg}
\setbeamercolor{palette sidebar quaternary}{use=structure,fg=structure.fg}
\setbeamercolor{palette sidebar secondary}{use=structure,fg=structure.fg}
\setbeamercolor{palette sidebar tertiary}{use=normal text,fg=normal text.fg}
\setbeamercolor{section in sidebar}{fg=brown}
\setbeamercolor{section in sidebar shaded}{fg=grey}
\setbeamercolor{separation line}{}
\setbeamercolor{sidebar}{bg=red}
\setbeamercolor{sidebar}{parent=palette primary}
\setbeamercolor{structure}{bg=white, fg=black}
\setbeamercolor{subsection in sidebar}{fg=brown}
\setbeamercolor{subsection in sidebar shaded}{fg=grey}
\setbeamercolor{title}{fg=brown}
\setbeamercolor{titlelike}{fg=brown}

%% \setbeamertemplate{background canvas}{\includegraphics
%% 	[width=\paperwidth,height=\paperheight]{alps.jpg}}

\newcommand{\code}[1]{\lstinline{#1}}

\usepackage[printwatermark]{xwatermark}
\usepackage{xcolor}
\usepackage{graphicx}
\usepackage{tikz}
\usepackage{lipsum}

% \newsavebox\mybox
% \savebox\mybox{\tikz[color=red,opacity=0.3]\node{DRAFT};}
% \newwatermark*[
%   allpages,
%   angle=45,
%   scale=6,
%   xpos=-20,
%   ypos=15
% ]{\usebox\mybox}

\begin{document}

\begin{frame}
\titlepage
\end{frame}

\begin{frame}{About Me}

\begin{itemize}
\item Founded the SMC project in 2001 while studying at REC Calicut
\item Employed by FSF India in 2002-2003
\item Contributor to Zope project (2006-2010)
\item Book Author: A Comprehensive Guide to Zope Component Architecture
\item During PyCon India 2013, received the first Kenneth Gonsalves Award
\end{itemize}

\end{frame}

\begin{frame}{Agenda}
\begin{itemize}
\item Installing Python
\item Setting up development environment
\item Distribution using Setuptools
\item MANIFEST.in
\item Console Scripts
\end{itemize}
\end{frame}

\begin{frame}{Installing Python}

\begin{itemize}

\item Download
\item Extract
\item ./configure --prefix=/path/to/install
\item make
\item make install

\end{itemize}
\end{frame}


\begin{frame}{Setting up development environment}

\begin{itemize}

\item \url{http://www.pip-installer.org/en/latest/installing.html}
\item pip install virtualenv
\item virtualenv venv1
\item source venv1/bin/activate

\end{itemize}
\end{frame}


\begin{frame}[fragile]
\frametitle{Distribution using Setuptools}

\small{
\begin{verbatim}
mkdir -p hello/hello
touch hello/hello/__init__.py
touch hello/hello/main.py
touch hello/setup.py
touch README.txt
\end{verbatim}
}
\end{frame}

\begin{frame}[fragile]
\frametitle{Distribution using Setuptools - Continued}
\small{
\begin{verbatim}
from setuptools import setup, find_packages

setup(name='hello',
      version='0.1.0',
      description='Hello World!',
      author='Baiju Muthukadan',
      packages=find_packages(),
      include_package_data=True,
      zip_safe=False,
      install_requires=[
          'setuptools',
      ],
      )
\end{verbatim}
}
\end{frame}

\begin{frame}[fragile]
\frametitle{Distribution using Setuptools - Continued}

Content of hello/hello/main.py:

\small{
\begin{verbatim}
def say_hello():
    return 'Hello, World!'
\end{verbatim}
}

\end{frame}

\begin{frame}[fragile]
\frametitle{Distribution using Setuptools - Continued}

Creating source distributions:

\small{
\begin{verbatim}
cd hello
python setup.py sdist
\end{verbatim}
}

Installing distributions:

\small{
\begin{verbatim}
pip install hello-0.1.0.tar.gz
\end{verbatim}
}

\end{frame}

\begin{frame}[fragile]
\frametitle{MANIFEST.in}

Content of MANIFEST.in:

\small{
\begin{verbatim}
include *.txt
\end{verbatim}
}

\end{frame}

\begin{frame}[fragile]
\frametitle{Console Scripts}

\small{
\begin{verbatim}
from setuptools import setup, find_packages

setup(name='hello',
      version='0.1.0',
      description='Hello World!',
      author='Baiju Muthukadan',
      packages=find_packages(),
      include_package_data=True,
      entry_points={
        'console_scripts': ['hello=hello.main:run',],
      },
      zip_safe=False,
      install_requires=[
          'setuptools',
      ],
      )
\end{verbatim}
}

\end{frame}

\begin{frame}[fragile]
\frametitle{Console Scripts - Continued}

Content of hello/hello/main.py:

\small{
\begin{verbatim}

def say_hello():
    return 'Hello, World!'

def run():
    hello = say_hello()
    print(hello)

\end{verbatim}
}

\end{frame}

\begin{frame}{Thanks!}
\center \url{http://muthukadan.net}
\end{frame}

\begin{frame}[fragile]
\frametitle{Command line parsing}

Content of hello/hello/main.py:

\small{
\begin{verbatim}
import argparse

def say_hello(name):
    return 'Hello, %s!' % name

def run():
    parser = argparse.ArgumentParser()
    parser.add_argument("name", help="Your name")
    args = parser.parse_args()

    hello = say_hello(args.name)
    print(hello)

\end{verbatim}
}

\end{frame}

\begin{frame}[fragile]
\frametitle{Command line parsing - Continued}

Content of hello/hello/main.py:

\small{
\begin{verbatim}
import argparse

def say_hello(name, count):
    hello_list = []
    for i in range(count):
        hello_list.append('Hello, %s!' % name)
    return hello_list

def run():
    parser = argparse.ArgumentParser()
    parser.add_argument("name", help="Your name")
    parser.add_argument("--count", type=int)
    args = parser.parse_args()

    for hello in say_hello(args.name, args.count):
        print(hello)

\end{verbatim}
}

\end{frame}


\end{document}
